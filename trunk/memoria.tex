\documentclass[12pt,spanish,letterpaper]{uchile}
\usepackage[ansinew]{inputenc}
\usepackage[spanish,activeacute]{babel}
%move to class?
%%usa punto en vez de coma en las ecuaciones con decimales, tiene que ir despues de babel
\decimalpoint
%%%%%%%%%%%%%%%%%%%%%%
%% Mathematical packages
\usepackage{amsmath}
\usepackage{amsthm}
\usepackage{amssymb}

%move to class?
%%  paquete para setear los margenes, verificar
\usepackage[right=2cm,left=3cm,top=2cm,bottom=2cm,headsep=0cm,footskip=1cm]{geometry}
%\usepackage{anysize}
%\papersize{27.94cm}{21.59cm}
%\marginsize{3.0cm}{2.0cm}{2.0cm}{2.0cm}


%move to class?
%% paquete para insertar links en pdf
%colorlinks=true, resalta los links con colores
\usepackage[pdftex,pdfborderstyle={/S/U/W 0}, 
pdftitle={Ingresar titulo}, 
pdfauthor={Ingresar autor}, 
pdfsubject={Ingresar Topico}, pdfkeywords={palabra clave 1, palabra clave 2},
pdfcreator={}]{hyperref}

%%%%%%%%%%%%%%%%%%%%%%
%% Cite Package
%% must go after hyperref
%% debe ir depues de hyperref
\usepackage{cite}

%%%%%%%%%%%%%%%%%%%%%%
%% Numprint Pacakge
%% allows to print number with automatic thousand separator and decimal separator
%% Ejxamples: \numprint{2006.3}-> 2.006,3 \numprint{20009}-> 20.009
%% permite escribir numeros con separador de miles y con separador decimal. 
%% Ejemplos: \numprint{2006.3}-> 2.006,3 \numprint{20009}-> 20.009
\usepackage[autolanguage]{numprint}
\npthousandsep{.}\npdecimalsign{,}

\newtheorem{lemma}{Lema}%we can specify the counter in brackets
\newtheorem{corollary}{Corolario}
\newtheorem{theorem}{Teorema}
%\renewcommand{\bibliographyname}{Bibliograf\'ia u otro nombre}
\begin{document}
%%%%%%%%%%%%%%%%%%%%
%Definicion de variables de la portada
%%%%%%%%%%%%%%%%%%%%
%% B. Nombre Institucion
\universidad{Universidad de Chile}
\facultad{Facultad de Ciencias F\'isicas y Matem\'aticas}
\departamento{Departamento de Ciencias de la Computaci\'on}
%% C. Titulo
\title{Sanidad de Rutas Chilenas en Internet}
%% D. Proposito titulacion
\trabajoygrado{Memoria para optar al t\'itulo de Ingeniero Civil en Computaci\'on}
%% E. Autor
\author{Pablo Ignacio Sep\'ulveda Rojas}
%% F. Profesores guia, co-guia e integrantes, los 3 primeros son obligatorios
\profguia{Sr. Tom\'as Barros} %profesor guia
\profcoguia{Sr. } %profesor co-guia
\profint{Sr. } %profesor integrante
%\profinta{Sr. ZZ ZZ ZZ} %profesor integrante 2, generalemente no es necesario
%\profintb{Sr. ZZZ ZZZ ZZZ} %profesor integrante 3, generalmente no es necesario
%% G. Lugar y fecha
\ciudad{Santiago}
\pais{Chile}
\monthpub{Mayo}
\yearpub{2009}
%por ahora hay que poner el proyecto en mayusculas y si los saltos de linea deben ir a mano
%\proyecto{ESTE TRABAJO HA SIDO FINANCIADO EN PARTE POR EL PROYECTO \\ FONDECYT 12345678 Y POR EL PROYECTO CORFO 2009/8956-4}
%%%%%%%%%%%%%%%%%%%%%
%% Lista de TODOS y FIXMES, no aparece si es que no hay nada que hacer
%\listoftodos
\newpage
%% Portada
\maketitle

%% Pagina optativa, ahora creo que ya no va
%\section{Calificaciones}
%\makeeval
% Necesita revision en caso de uso
\begin{preface}
%%%%%%%%%%%%%%%%%%%%%%%%%%%%%%
%% Resumen
%\section{Resumen}

%%%%%%%%%%%%%%%%%%%%%%%%%%%%
%% Dedicatoria: Pagina Optativa
%\dedicatoria{A mis compa�eros ingenieros}
%% Pagina optativa
%%%%%%%%%%%%%%%%%%%%%%%%%%%%
%% Agradecimientos: Pagina Optativa
%\section{Agradecimientos}
%Se agradece a 
%%%%%%%%%%%%%%%%%%%%%%%%%%%%%%
%% Indices varios
%% Indice General
\tableofcontents
%% Indice de Tablas : Pagina Optativa
%\listoftables
%% Indice de Figuras : Pagina Optativa
%\listoffigures
\end{preface}
%%%%%%%%%%%%%%%%%%%%%%%%%%%%%%
\chapter{Introducci\'on}
Internet, llamada la red de redes, es una red global descentralizada, compuesta
de muchas redes m\'as peque\~nas interconectadas. Una red que est\'e bajo el 
control de una sola organizaci\'on, se llama Sistema Aut\'onomo (Autonomous 
System, AS). Internet est\'a formado por la uni\'on de estos ASes.

El proceso de ruteo dentro de un AS es llamado ruteo intra-dominio y el
ruteo entre distintos ASes se conoce como ruteo inter-dominios. El ruteo 
inter-dominio usado en internet es el Border Gateway Protocol ~\cite{BGP}
(BGP). Aunque su simplicidad y elasticidad le han permitido tener el rol 
principal en la Internet global, hist\'oricamente ha tenido muy pocas 
garant\'ias de seguridad, pues su dise\~no original no lo consider\'o 
~\cite{bgp-eye}.
	
Debido a falta de seguridad y al aspecto inter-dominio de BGP, incluso 
peque\~nas fallas dentro de un AS pueden, en ocasiones, propagarse ampliamente 
al resto de Internet, causando da\~no masivo. Una de estas fallas, ocurri\'o en
abril de 1997, cuando un router mal configurado de un peque\~no proveedor de
acceso a internet (Internet Service Provider, ISP) insert\'o informaci\'on
err\'onea en Internet global, diciendo que ten\'ia una ruta \'optima a todos
los destinos ~\cite{bgp-eye}. Como no hay una forma de validar los anuncios de 
ruta, la mayor parte del tr\'afico de Internet fue redirigida a esta ISP, lo que
da\~n\'o fuertemente Internet por casi dos horas. La p\'erdida de
accesibilidad en Internet, puede deberse a errores humanos, por
ejemplo: mala configuraci\'on de un router, o actividades maliciosas.

Otro problema m\'as reciente fue el llamado Hijack de
Youtube~\cite{pakistan-youtube}. El 24 de febrero de 2008, Pakist\'an Telecom 
comenz\'o a avisar que cierta parte de la red de Youtube le pertenec\'ia. 
Esto ocasion\'o que cerca de dos tercios de los usurios de Internet que 
quer\'ian acceder a esa parte de la red, fueran redirigidos hacia Pakist\'an
Telecom, dej\'andolos sin poder disfrutar de sus videos. Este ``ataque''
fue hecho intencionalmente, ya que el gobierno hab\'ia ordenado bloquear el 
acceso a Youtube. Gracias a la r\'apida reacci\'on del equipo de Youtube, y a 
la cooperaci\'on de otros proveedores,  el problema dur\'o s\'olo un par de 
horas.
		
BGP ha sido el protocolo de ruteo en Internet por m\'as de 10 a\~nos, durante
los cuales numerosos problemas han sido descubiertos. Los administradores de
cada red pueden proteger sus redes de la mayor parte de estos problemas usando 
autentificaci\'on, y otras buenas pr\'acticas. Sin embargo no es posible 
verificar f\'acilmente el contenido de un mensaje BGP (la verificaci\'on 
completa toma demasiado tiempo)~\cite{BGP-path-stability}, por defecto un 
router cree la informaci\'on que recibe. Para enfrentar esta situaci\'on, se 
cre\'o Pretty Good BGP~\cite{pgbgp0, pgbgp1, pgbgp2} (PGBGP) un sistema que no conf\'ia autom\'aticamente en las rutas, sino que trata de dar tiempo para saber si son reales antes de anunciarlas c\'omo ruta \'optima.
			
Todos estos problemas de Internet, ocasionan demoras en el tr\'afico e incluso
p\'erdida de alcance de algunos destinos. En este trabajo se estudiar\'a hasta 
qu\'e punto es efectivamente alcanzable (o visible) la red chilena desde el 
mundo.

El protocolo de comunicaci\'on usado actualmente es Internet Protocol (IP), ya 
sea en su cuarta o sexta versi\'on (IPv4 e IPv6). Cada computador debe tener un
n\'umero IP \'unico que lo identifique dentro de Internet. Los n\'umeros IP son
vendidos a organizaciones y a los ISP (por ejemplo: VTR, GTD, Telef\'onica) 
para que \'estos puedan administrarlos dentro de su red interna. Es as\'i como 
se puede obtener una aproximaci\'on sobre cual es el rango de IP que se 
encuentra en Chile; para de esta forma hacer el estudio centrado 
espec\'ificamente en nuestro pa\'is. Determinar la localizaci\'on geogr\'afica
a partir de la IP no es trivial para el caso general, ver ~\cite{caida-asia}.
				
Esta memoria, pretende por un lado obtener informaci\'on acerca de la 
conectividad actual de Internet en Chile, a nivel de IP y de AS y por otro lado
probar una estrategia para mejorar la seguridad de BGP, en particular PGBGP.
%
%\newpage
\section{Motivaci\'on}
Internet es el fen\'omeno tecnol\'ogico, econ\'omico y social m\'as importante 
de nuestros tiempos. Es un medio de comunicaci\'on, acceso y distribuci\'on de 
la informaci\'on que no tiene  comparaci\'on con ning\'un otro medio inventado 
por el hombre.

En Chile se sabe poco sobre la calidad de nuestra Internet, �Es visible mi
sitio desde Europa?, actualmente se puede responder a esta pregunta, pues
existen servidores que informan si cierta IP es visible (looking glasses),
sin embargo hasta ahora estos servicios s\'olo consideran una IP y un momento
dado, y no el rango completo de las IPs chilenas, ni su variabilidad en el
tiempo, que es lo que se intentar\'a lograr en esta memoria.

Es interesante entonces saber qu\'e porcentaje de todas las IPs de Chile son 
visibles desde el mundo, como una medida de la visibilidad de nuestra red; es 
decir hacer el estudio para un todas las IPs de Chile (i.e. los AS), con la 
finalidad de tener datos agregados a nivel pa\'is, que permitir\'ian conocer la
realidad chilena, respecto a otros pa\'ises, en esta tem\'atica.

En esta misma l\'inea, es relevante saber si el problema de la visibilidad es, 
por ejemplo, responsabilidad de alg\'un ISP en particular, es decir tratar de 
aislar lo mejor posible el problema para notificar a los involucrados (de forma
autom\'atica) de modo que el problema pueda ser estudiado y en lo posible 
solucionado por los responsables espec\'ificos.

Es fundamental el saber qu\'e tanto se puede mejorar la seguridad de la red en 
Chile, pues muchas de las actividades realizadas en el pa\'is dependen de 
Internet, por lo tanto es cr\'itico que \'esta tenga alg\'un nivel de
estabilidad, o seguridad.

Cooperative Association for Internet Data Analysis (CAIDA), es una asociaci\'on
que provee herramientas y an\'alisis, promoviendo el mantenimiento de una 
infraestructura de Internet escalable y robusta. Dentro de esos an\'alisis ha
hecho estudios en el \'area de la visibilidad ~\cite{caida-asia}, para la zona
Asia-Pac\'ifico, en este estudio, no se incluy\'o Chile, pero s\'olo con 28
destinos en el pa\'is, por lo tanto es beneficioso centrar un estudio similar
espec\'ificamente en este pa\'is.

La memoria pretende investigar la visibilidad y el crecimiento de Internet en 
Chile, qu\'e IPs son compradas en Chile y cu\'ales son efectivamente visibles. 
De este modo se pretende obtener no solo una fotograf\'ia de internet, sino que
adem\'as poder ir componiendo estas fotograf\'ias en el tiempo para ver su 
evoluci\'on temporal.
Acad\'emicamente es muy interesante saber esto, pues actualmente s\'olo se han 
realizado fotos, y para saber c\'omo es Internet hoy en Chile, habr\'ia que 
rehacer el estudio. Estos datos pueden ser usados para ver como migra Chile 
realmente hacia IPv6, un paso que debe tomarse ( y efectivamente est\'a siendo 
tomado).

\section{Objetivo General}
El objetivo general es medir la visibilidad de Internet en Chile en el tiempo,
as\'i como su crecimiento y seguridad ante eventuales amenazas.
\section{Ojetivos Espec\'ificos}
\begin{itemize}
 \item Crear (o adaptar) una herramienta que permita medir la visibilidad de
 las IPs y  ASes chilenas en los ``looking glasses''.
 \item Realizar esta  medici\'on de forma constante, y publicar la 
 informaci\'on para el p\'ublico.
 \item Medir crecimiento de Internet en Chile (IPs compradas y  visibles)
 \item La herramienta debe permitir detectar problemas y puntos de falla.
 \item Estudiar y probar PGBGP. Recomendar (o no) su uso, considerando en esto,
 tanto de su utilidad al evitar falsas rutas, como el retraso provocado por su
 uso.
\end{itemize}
\section{Metodolog\'ia}
\begin{enumerate}
 \item Se revisar\'an protocolos para conectarse a los ``looking glasses''
 existentes, de manera de crear (o adaptar) un cliente que pruebe con las IPs
 pertenecientes a Chile, as\'i como con los AS chilenos.
 \item En una primera etapa, se busca que se pueda probar con alguna IP, y
 se obtenga el porcentaje de \'exito con respecto a al menos 50 ``looking
 glasses''
 \item Dentro de esta etapa se mostrar\'a tambie\'en los puntos de falla.
 \item En una segunda etapa se espera poder realizar autom\'aticamente la 
 prueba para todas las IPs de Chile.
 \item Se obtendr\'an datos con el objetivo de generar informaci\'on, es decir,
 porcentaje de visibilidad, un mapa de visibilidad que contenga: punto de
 vista, lugar  observado.
 \item Se pretende usar esta herramienta para buscar los problemas que existan
 actualmente (si un ISP rutea mal por ejemplo).
 \item Se estudiar\'a BGP y PGBGP se realizar\'a una implementaci\'on de este
 \'ultimo.
 \item Se probar\'a experimentalmente la utilidad de este sistema.
\end{enumerate}


\chapter{Trabajo Relacionado}
El primer estudio completo de la sanidad de internet fue hecho en 1997
~\cite{instability}. All\'i se observ\'o por nueve meses el tr\'afico a
trav\'es de 5 puntos de intercambio, sus principales descubrimientos fueron la
gran cantidad de actualizaciones que se realizaban, la mayor parte patol\'ogica
o redundante. Casi diez a\~nos despu\'es, un estudio similar, 
~\cite{routing-revisited} con datos desde agosto 2005 a enero 2006, mostrando 
una clara mejora en la salud de BGP, los mensajes patol\'ogicos disminuyeron 
notablemente su presencia, quedando entre estos, mayoritariamente los anuncios 
redundantes.

Existen iniciativas p\'ublicas para almacenar la informaci\'on de ruteo, entre
ellas est\'an los archivos mantenidos por el Proyecto Vista de Rutas de la
Universidad de Oregon (University of Oregon's Route Views Project)
~\cite{route-views}, los archivos RIPE ~\cite{ripe} y el Monitor BGP del MIT
~\cite{bgp-monitor}, estos archivos almacenan la informaci\'on tal como se
origin\'o, es decir los mensajes de actualizaci\'on, por lo que no es sencillo
obtener informaci\'on o estad\'isticas de ellos. BGP-Play ~\cite{bgp-play} es
una herramienta que permite visualizar el grafo de ASes al igual que LinkRank
~\cite{link-rank}, por otro lado, BGP-Inspect ~\cite{bgp-inspect} es una 
herramienta que obtiene y preprocesa informaci\'on de estos archivos, 
proporcionando una interfaz de consulta web m\'as simple desde la cual se puede
obtener informaci\'on agregada.

Adem\'as de estos sitios, existen actualmente servidores ``looking glasses,''
que permiten obtener informaci\'on de ruteo, desde ese punto hasta cualquier
otro punto~\cite{looking-glasses}, ingresando la IP del destino.

Otra iniciativa, es la de CAIDA, que sit\'ua servidores en distintas partes del
mundo para tener mediciones desde distintos puntos de vista. Esta
organizaci\'on realiz\'o un estudio para la zona de Asia y el Pac\'ifico,
incluyendo 28 destinos dentro de Chile, un estudio similar es el que se
pretende realizar en esta memoria, pero centrado exclusivamente en Chile como
destino, y con la mayor cantidad de lugares de origen posible.

Con respecto a la seguridad de BGP, se han tomado dos enfoques:
\begin{enumerate}
\item Protecci\'on criptogr\'afica
\item Detecci\'on de anomal\'ias
\end{enumerate}
Protecci\'on criptogr\'afica se relaciona con un registro autentificado que
relacione prefijos IP con sus respectivos ASes de origen. El registr\'o
deber\'a ser firmado y distribuido usando criptograf\'ia de clave asim\'etrica,
o de clave p\'ublica (PKI por su sigla en ingl\'es).

Secure BGP (SBGP)~\cite{sbgp} fue el primer intento, en que cada ruta anunciada
es firmada por el AS de origen para asegurar que el ruteador representa el ASN
que dice tener, y que el prefijo pertenece al ASN. Los ASes que propagan la
ruta, deben firmar tambi\'en, para verificar que el camino es el que realmente
recorri\'o la actualizaci\'on.

Secure Origin BGP (soBGP) ~\cite{sobgp} toma un enfoque m\'as descentralizado.
Usa un PKI distribuido que contiene la propiedad de los prefijos y objetos de
pol\'iticas. Los objetos de pol\'iticas son usados para declarar que dos ASes
son vecinos, y los prefijos que deber\'ian ser propagados en ese v\'ertice.
SoBGP verifica una actualizaci\'on, primero asegur\'andose que el AS de origen
del camino es correcto usando una b\'usqueda PKI, y luego verifica que el
camino de AS es posible dada la secuencia de aristas para el prefijo y los
objetos de pol\'iticas.

Pretty Secure BGP (psBGP) ~\cite{psbgp} emplea un PKI centralizado, para
distribuir certificados de n\'umeros AS, y una red de confianza PKI para
designar due\~nos de prefijos.

Desafortunadamente, no es claro como pueden manejar caminos que no pueden ser
completamente verificados debido a que no todos los ASes han adoptado su
soluci\'on.

Entre los m\'etodos que detectan anomal\'ias est\'a ~\cite{topology-anomaly},
que analizando los ``vecinos'' de cada AS detecta rutas falsas, aquellas cuyo
camino contiene ASes consecutivos que no son vecinos.

Otros trabajos enfocados en la  detecci\'on de anomal\'ias son ~\cite{whisper} 
y ~\cite{moas}, los cuales usan cierto atributo del anuncio BGP para transmitir
informaci\'on extra, \'util para sus fines, pero que probablemente ser\'a
descartada por routers que no tengan implementada su soluci\'on.

Finalmente Pretty Good BGP (PGBGP) ~\cite{pgbgp0, pgbgp1, pgbgp2} se basa en 
retrasar el momento en que un anuncio es diseminado, baj\'andole la prioridad a
los anuncios nuevos que recibe, hasta que haya pasado un tiempo razonable para 
que el administrador se entere que la ruta era inv\'alida. Esto no es lo mismo 
que ignorar todos los anuncios, pues solamente se cambia la prioridad, si la 
ruta es la \'unica existente para ese destino, si ser\'a utilizada. Seg\'un sus
propios experimentos, bastaba con que un porcentaje menor de los ASes
implementara la soluci\'on para lograr una gran disminuci\'on de los routers
que cre\'ian las rutas falsas.
%% PGBGP
%%CODIGO:
%http://www.cs.unm.edu/~treport/tr/07-08/pgbgp-adversary.pdf
%%
%

%%%%%%%%%%%%%%%%%%%%%%%%%%%%%%
%\chapter{Conceptos B\'asicos}
%\texttt{\textbackslash numprint\{20009\}}: \numprint{20009}\\

%Opciones de clase:
%\begin{itemize}
%\item \textbf{color=default:} inserta logo a color
%\item \textbf{bw:} inserta logo en blanco y negro
%\item \textbf{singlespace:} usa espacio simple (mata portada)
%\item \textbf{doublespace:} usa espacio doble (mata portada)
%\item \textbf{defaultspace=default:}usa espacio y medio
%\end{itemize}

%Se definieron los siguientes comandos:
%\begin{itemize}
%\item \textbf{\textbackslash TODO\{text\}} agrega un todo
%\item \textbf{\textbackslash FIXME\{text\}} agrega un fixme
%\item \textbf{\textbackslash bibliographyname} define el nombre de la bibliografia por defecto a referencias, se puede cambiar con \textbackslash renewcommand\{\textbackslash  bibliographyname\}\{text\}
%\end{itemize}

%Recomendaci\'on, generar siempre las imagenes en pdf. El logo a color esta generado en base al logo vectorial hecho por mechon barsa y B1mbo, sacado de http://es.wikipedia.org/wiki/Archivo:Uchile.svg, aunque le tuve que sacar la capa del brillo, porque me generaba mal el pdf. Ahora me faltaria generar el pdf del logo en blanco y negro.

%Las imagenes las pueden ubicar en la carpeta img para mas orden, y no es necesario ponerle la ruta
%por ejemplo el archivo a.pdf esta en img/, y el archivo b.pdf esta en img/algo, y para insertarlo solo pongo
%\texttt{a.pdf}, o \texttt{algo/b.pdf}. Tambien se pueden omitir las extensiones, y primero busca por .pdf, luego por .png y luego por .jpeg o.jpg, pero es mejor insertar pdf.
%\includegraphics[height=2cm]{a.pdf}
%\includegraphics[height=2cm]{algo/b.pdf}
%\includegraphics[height=2cm]{algo/b}
%\includegraphics[height=2cm]{a.pdf}
%%%%%%%%%%%%%%%%%%%%%%%%%%%%%%
%\chapter{Evaluaci\'on Experimental}
%Este template ha sido utilizado tanto en las memorias de Fernando Krell Loy y Sebastian Kreft Carre�o. Tambi\'en estudiantes de Ingenieria Civil Industrial y El\'ectrica han usado versiones antiguas de este template.
%%%%%%%%%%%%%%%%%%%%%%%%%%%%%%
%\chapter{Conclusiones}
%Se obtuvo un template que permite escribir r\'apidamente una memoria en \LaTeX  sin las complicaciones de tener que hacer la portada ni de crear la estructura.
%%%%%%%%%%%%%%%%%%%%%%%%%%%%%%
\bibliographystyle{abbrv}
\bibliography{biblio}{}
%\begin{additional}
%\section{Notas aclaratorias}
%Este template
%\section{Glosario}
%\section{Material Complementario}
%\end{additional}

%\begin{appendix}
%\section{C\'odigo 1}
%Aqu\'i debiese ir el c\'odigo
%\newpage
%\section{Art\'iculo publicado}
%Y ac\'a algun articulo
%\end{appendix}
\end{document}
